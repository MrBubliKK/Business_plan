 \section{Финснсовый план}

\subsection{Прогноз объемов реализации}

Основываясь на расчетах плана производства и реализации продукции, прогнозы объемов реализации ООО "Astadeer" выглядят следующим образом:

Прогноз продаж на первый квартал:

\begin{itemize}
    \item Месяц 1: 350 единиц продукции
    \item Месяц 2: 420 единиц продукции (+20\% к первому месяцу)
    \item Месяц 3: 525 единиц продукции (+25\% ко второму месяцу)
\end{itemize}
Итого за квартал: 1295 единиц продукции

\subsection{Расчёт себестоимости:}

Себестоимость персонализированной продукции будет рассчитываться на основе \textbf{прямых} и \textbf{косвенных затрат}.

\textbf{Прямые статьи себестоимости:}

\begin{itemize}[noitemsep]
    \item Стоимость базовых изделий (футболки, толстовки).
    \item Расходные материалы для печати (чернила, пленки и т.д.).
    \item Упаковочные материалы.
    \item Оплата труда производственного персонала (сдельная оплата).
\end{itemize}

\textbf{Косвенные статьи себестоимости:}
\begin{itemize}[noitemsep]
\item Аренда производственного помещения (если есть).
\item Коммунальные платежи (электроэнергия, отопление и т.д.).
\item Заработная плата управленческого и административного персонала.
\item Маркетинговые расходы.
\item Расходы на обслуживание онлайн-платформы (хостинг, поддержка).
\item Амортизация оборудования.
\item Прочие административные и операционные расходы.
\end{itemize}

\subsubsection{Расчета затрат, цены, себестоимости для футболки}

\textbf{расчета прямых затрат на футболку:}

\begin{itemize}[noitemsep]
    \item Стоимость футболки (закупка): 20 BYN
    \item Расходные материалы для печати (в среднем): 5 BYN
    \item Упаковка: 1 BYN
    \item Оплата труда печатника (за единицу): 4 BYN
\end{itemize}

\textbf{Итого прямые затраты: 30 BYN}

\textbf{Расчета себестоимости и цены:}

\begin{itemize}[noitemsep]
\item Прямые затраты на футболку: 30 BYN
\item Косвенные затраты (распределенные на единицу продукции): 10 BYN
\item \textbf{Полная себестоимость: 40 BYN}
\item Планируемая наценка (уровень прибыли) - 50\%
\item \textbf{Цена (затратный метод): 40 BYN + (40 BYN * 0.5) = 60 BYN}
\end{itemize}

\subsubsection{Расчета затрат, цены, себестоимости для толстовки}

\textbf{расчета прямых затрат на толстовку:}

\begin{itemize}[noitemsep]
    \item Стоимость толстовки (закупка): 30 BYN
    \item Расходные материалы для печати (в среднем): 10 BYN
    \item Упаковка: 2 BYN
    \item Оплата труда печатника (за единицу): 8 BYN
\end{itemize}

\textbf{Итого прямые затраты: 50 BYN}

\textbf{Расчета себестоимости и цены:}

\begin{itemize}[noitemsep]
\item Прямые затраты на толстовку: 50 BYN
\item Косвенные затраты (распределенные на единицу продукции): 10 BYN
\item \textbf{Полная себестоимость: 60  BYN}
\item Планируемая наценка (уровень прибыли) - 50\%
\item \textbf{Цена (затратный метод): 60 BYN + (60 BYN * 0.5) = 90 BYN}
\end{itemize}


Затратным методом определена цена за единицу: 60 BYN (футболки), 90 BYN (толстовки). Для расчётов используется усреднённая цена — 75 BYN/ед. 
Себестоимость: 40 BYN (футболки), 60 BYN (толстовки). Средняя себестоимость 50 BYN.

Таким образом, прогнозируемая выручка на первый квартал составит:
\begin{equation}
1295 \text{ единиц } \times 75 \text{ BYN } = 97 125 \text{ BYN}
\end{equation}

Валовая прибыль за квартал:
\[
\text{ВП} = \text{Выручка} - \text{Себестоимость} = 97\,125 - (1295 \times 50) = 97\,125 - 64\,750 = 32\,375 \text{ BYN}
\]



\begin{table}[h!]
    \centering
    \caption{Прогноз объемов реализации}
    \begin{tabular}{|l|c|c|c|}
        \hline
        Показатель & 1-й месяц & 2-й месяц & 3-й месяц \\
        \hline
        Объём продаж, ед. & 350 & 420 & 525 \\
        Выручка, BYN & 26\,250 & 31\,500 & 39\,375 \\
        \hline
    \end{tabular}
\end{table}


\subsection{Расчет финансовых показателей}

\begin{itemize}
    \item Цена реализации: 75 BYN
    \item Себестоимость: 50 BYN
    \item Валовая прибыль на единицу: \( \text{ВП}_\text{ед} = P - C = 75 - 50 = 25 \, \text{BYN} \)
    \item Рентабельность единицы: \( R_\text{ед} = \frac{\text{ВП}_\text{ед}}{C} \cdot 100\% = \frac{25}{50} \cdot 100\% = 50\% \)
\end{itemize}

Прибыль в день (при производстве 25 единиц в день):
\begin{equation}
25 \text{ BYN } \times 25 \text{ единиц } = 6255 \text{ BYN}
\end{equation}

Прибыль в неделю (при 5 рабочих днях в неделе):
\begin{equation}
625 \text{ BYN } \times 5 \text{ дней } = 3 125\text{ BYN}
\end{equation}

Прибыль в месяц (4,3 рабочих недели):
\begin{equation}
3 125 \text{ BYN } \times 4,3 \text{ недель } = 13 437.5 \text{ BYN}
\end{equation}

Рентабельность продаж:
\begin{equation}
\left(\frac{\text{Прибыль}}{\text{Выручка}}\right) \times 100% = \left(\frac{40 \text{ BYN}}{80 \text{ BYN}}\right) \times 100% = 50%
\end{equation}

Срок окупаемости первоначальных инвестиций (при первоначальных инвестициях 205 000 BYN и месячной прибыли 21 500 BYN):
\begin{equation}
\frac{205 000 \text{ BYN}}{13 437.5 \text{ BYN}} = 15.2 \text{ месяцев}
\end{equation}

Таким образом, организация ООО "Astadeer" демонстрирует  рентабельность 50\% и относительно короткий срок окупаемости (около 15.2 месяцев), что свидетельствует о его финансовой привлекательности и жизнеспособности.

 Для более быстрого выхода на окупаемость требуется:
\begin{itemize}
    \item Увеличить объём продаж до 1\,000 ед./мес.
    \item Сократить постоянные расходы.
    \item Повысить цену реализации.
\end{itemize}




