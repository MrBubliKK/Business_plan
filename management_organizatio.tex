%% LaTeX2e file `management_organizatio.tex'
%% generated by the `filecontents' environment
%% from source `main' on 2025/04/08.
%%
    \section{Управление и организация}

    В данном разделе описывается организационная структура управления ООО "Astadeer", включая информацию об основных участниках предприятия и организационной схеме управления, выбранной для эффективной работы и достижения целей компании.

    \subsection{Основные участники предприятия}

    {\Large \textbf{Руководители компании:} \par} % Заголовок "Руководители компании"
    \vspace{0.2cm}

    \begin{center} % Окружение center для центрирования таблицы
    \begin{tabular}{ll} % Окружение tabular для таблицы
        \textbf{ФИО} & \textbf{Должность} \\ % Заголовки столбцов таблицы
        \hline
        Машкович Вячеслав Николаевич & Генеральный директор \\
        Глёза Егор Дмитриевич & Коммерческий директор \\
        Карпук Максим Витальевич & Финансовый директор \\
        Дичковский Владимир Андреевич & Технический директор \\
    \end{tabular}
    \end{center}
    \vspace{1.5cm}

    На начальном этапе, ключевые управленческие функции в ООО "Astadeer" распределены между указанными руководителями, каждый из которых обладает значительным опытом и компетенциями в своей области и отвечает за определенное направление деятельности, подчиняясь непосредственно Генеральному директору.

    \begin{itemize}
        \item \textbf{Генеральный директор (Директор): Машкович Вячеслав Николаевич} - основатель и высшее должностное лицо ООО "Astadeer", определяющий стратегическое видение и обеспечивающий общее руководство компанией.
        \item \textbf{Коммерческий директор: Глёза Егор Дмитриевич} - отвечает за коммерческий успех ООО "Astadeer", разрабатывая и реализуя стратегии для увеличения продаж, расширения рынка и повышения узнаваемости бренда.
        \item \textbf{Финансовый директор: Карпук Максим Витальевич} - обеспечивает финансовое благополучие и устойчивость ООО "Astadeer", управляя финансовыми потоками, планированием и отчетностью.
        \item \textbf{Технический директор: Дичковский Владимир Андреевич} - отвечает за технологическое превосходство и бесперебойную работу всех технических систем ООО "Astadeer", обеспечивая инновации и эффективность производственных и IT-процессов.
    \end{itemize}

    \subsection{Организационная схема управления}

    Для ООО "Astadeer" на начальном этапе развития наиболее подходящей является \textbf{линейно-функциональная организационная структура управления}, представленная в виде следующей схемы:

    \vspace{0.5cm}

    \begin{center}
    \begin{tikzpicture}[node distance=1.5cm,
      block/.style={rectangle, draw, text centered, rounded corners, minimum height=1cm, minimum width=2cm},
      level 1/.style={sibling distance=4cm}] % Removed level 2 style as it's not used
    \node[block] (CEO) {\textbf{Генеральный\\ директор}};
    \node[block, below left=of CEO] (CD) {\textbf{Коммерческий\\ директор}};
    \node[block, below=of CEO] (FD) {\textbf{Финансовый\\ директор}};
    \node[block, below right=of CEO] (TD) {\textbf{Технический\\ директор}};

    \draw [->] (CEO) -- (CD);
    \draw [->] (CEO) -- (FD);
    \draw [->] (CEO) -- (TD);

    \end{tikzpicture}
    \captionof{figure}{Линейно-функциональная организационная схема управления ООО "Astadeer" (начальный этап)}
    \end{center}

    \vspace{0.3cm}

    \textbf{Описание линейно-функциональной структуры:}

    Линейно-функциональная структура сочетает в себе элементы линейного и функционального управления.

    \begin{itemize}
        \item \textbf{Линейное управление:}  Генеральный директор осуществляет прямое руководство ключевыми специалистами (директорами направлений), выстраивая четкую вертикальную иерархию управления.  Решения и команды передаются по линейной цепочке сверху вниз.  Генеральный директор несет персональную ответственность за результаты деятельности компании в целом.
        \item \textbf{Функциональное управление:}  Коммерческий, Финансовый и Технический директора отвечают за определенные функциональные области и обладают определенной самостоятельностью в рамках своей компетенции.  Они могут принимать решения и руководить работой в своих направлениях, обеспечивая профессиональную экспертизу и специализацию.
    \end{itemize}

    \vspace{0.3cm}

    \textbf{Порядок подбора, подготовки и оплаты труда сотрудников:}

    \textbf{Подбор персонала:}

    Подбор персонала в ООО "Astadeer" осуществляется на конкурсной основе, с учетом требований к квалификации, опыту работы и личным качествам кандидатов.  \textbf{Основные этапы подбора персонала включают:}

    \begin{enumerate}
        \item \textbf{Определение потребности в персонале и формирование требований к кандидатам:}  Определение необходимого количества сотрудников, должностных обязанностей, квалификационных требований, уровня образования и опыта работы, а также ключевых личных качеств.
        \item \textbf{Размещение объявлений о вакансиях:}  Размещение объявлений о вакансиях на специализированных онлайн-платформах по поиску работы, в социальных сетях, на сайте компании (после запуска) и в других рекрутинговых каналах.
        \item \textbf{Отбор резюме и проведение предварительных собеседований:}  Анализ полученных резюме на соответствие требованиям вакансии, отбор наиболее подходящих кандидатов и проведение телефонных или онлайн-собеседований для предварительной оценки.
        \item \textbf{Проведение очных собеседований и тестирования:}  Организация очных собеседований с отобранными кандидатами, проведение профессионального тестирования (при необходимости) и оценка соответствия кандидатов требованиям вакансии и корпоративной культуре компании.
        \item \textbf{Принятие решения о найме и оформление трудовых отношений:}  Выбор наиболее подходящего кандидата на основе результатов собеседований и тестирования, согласование условий трудового договора и оформление приема на работу в соответствии с трудовым законодательством Республики Беларусь.
    \end{enumerate}

    \textbf{Подготовка и адаптация персонала:}

    Новые сотрудники ООО "Astadeer" проходят \textbf{процесс адаптации и вводного обучения}, который включает:

    \begin{itemize}
        \item \textbf{Ознакомление с компанией и корпоративной культурой:}  Вводная лекция или инструктаж о компании, ее истории, миссии, ценностях, организационной структуре и корпоративной культуре.
        \item \textbf{Обучение стандартам и технологиям работы:}  Инструктаж по должностным обязанностям, стандартам работы, технологическим процессам и используемому оборудованию и программному обеспечению.  Практическое обучение на рабочем месте под руководством опытного наставника.
        \item \textbf{Инструктаж по охране труда и технике безопасности:}  Обязательный инструктаж по правилам охраны труда, техники безопасности, пожарной безопасности и другим нормативным требованиям, связанным с рабочим местом и выполняемыми обязанностями.
        \item \textbf{Период испытательного срока:}  Установление испытательного срока \textbf{[Указать продолжительность испытательного срока, например, 3 месяца]} для оценки профессиональной пригодности сотрудника и его адаптации в коллективе.  В период испытательного срока сотрудник получает поддержку и обратную связь от руководителя и наставника.
    \end{itemize}

    \textbf{Оплата труда сотрудников:}

    Система оплаты труда в ООО "Astadeer" планируется комбинированной, включающей \textbf{окладную (повременную) и премиальную (сдельную) части}, в зависимости от должности и функциональных обязанностей сотрудника.

    \begin{itemize}
        \item \textbf{Руководители компании (Генеральный, Коммерческий, Финансовый, Технический директора):}  \textbf{Окладная система оплаты труда} с ежемесячным фиксированным окладом, размер которого определяется на основе рыночного уровня заработной платы для аналогичных должностей и опыта работы.  Возможно \textbf{премирование по итогам работы за квартал или год} за достижение ключевых показателей эффективности (KPI) и выполнение стратегических целей компании.  \textbf{Размер заработной платы руководителей: [Указать диапазоны заработной платы для руководителей, например, от 3000 BYN до 7000 BYN в зависимости от должности и опыта]}.
        \item \textbf{Дизайнер-технолог, Операторы печатного оборудования, Упаковщик-комплектовщик:}  \textbf{Повременно-сдельная система оплаты труда}, включающая \textbf{фиксированный оклад} (меньшую часть заработной платы) и \textbf{сдельную премию}, зависящую от объема и качества выполненной работы (количество произведенных изделий, выполненных заказов, отсутствие брака и т.д.).  Сдельная часть стимулирует производительность и качество работы персонала.  \textbf{Размер заработной платы производственного персонала: [Указать диапазоны заработной платы для производственного персонала, например, от 1000 BYN до 2000 BYN, в зависимости от должности, квалификации и выработки]}.
        \item \textbf{Уборщик производственных помещений:}  \textbf{Повременно-премиальная система оплаты труда} с почасовой оплатой или фиксированным окладом за частичную занятость и \textbf{премией за качественное и своевременное выполнение обязанностей}.  \textbf{Размер заработной платы уборщика: [Указать размер заработной платы или почасовую ставку]}.
    \end{itemize}

    \textbf{Заработная плата сотрудников ООО "Astadeer" планируется на уровне [Указать уровень заработной платы относительно рынка, например,  среднерыночном или выше среднего по рынку для аналогичных должностей в г. Минске], что позволит привлекать и удерживать квалифицированный и мотивированный персонал.}  Конкретные размеры окладов, сдельных расценок и премиальных выплат будут разработаны и утверждены в Положении об оплате труда ООО "Astadeer" и трудовых договорах с сотрудниками.

    \subsection{Организация учета и контроля выполнения планов, мотивация}

    \textbf{Организация учета и контроля выполнения планов:}

    В ООО "Astadeer" планируется внедрить \textbf{систему оперативного учета и контроля выполнения производственных и финансовых планов}, основанную на следующих принципах:

    \begin{itemize}
        \item \textbf{Регулярный сбор и анализ данных:}  Ежедневный учет количества принятых заказов, произведенной и отгруженной продукции, объема продаж и финансовых показателей (выручка, себестоимость, прибыль и т.д.).  Еженедельный и ежемесячный анализ собранных данных для оценки выполнения планов и выявления отклонений.
        \item \textbf{Использование информационных технологий:}  Внедрение \textbf{простой системы учета заказов и складского учета} на базе Google Sheets или Excel на начальном этапе, с перспективой перехода на специализированное \textbf{программное обеспечение для учета и CRM} по мере роста бизнеса.  Использование IT-систем для автоматизации сбора и обработки данных, формирования отчетов и визуализации ключевых показателей.
        \item \textbf{Установление ключевых показателей эффективности (KPI):}  Определение KPI для каждого подразделения и сотрудника, связанных с выполнением планов производства, продаж, финансовых показателей и качества работы.  KPI должны быть измеримыми, достижимыми, релевантными и ограниченными по времени (SMART-критерии).
        \item \textbf{Регулярные совещания и обсуждение результатов:}  Еженедельные или ежемесячные совещания руководителей подразделений с Генеральным директором для обсуждения результатов работы, анализа отклонений от планов, выявления проблем и принятия корректирующих мер.  Оперативное обсуждение текущих вопросов и проблем в рамках подразделений.
        \item \textbf{Система отчетности:}  Разработка системы регулярной отчетности для всех уровней управления, обеспечивающей своевременное предоставление информации о выполнении планов, финансовых показателях, проблемах и рисках.  Формирование аналитических отчетов и дашбордов для визуализации ключевых данных.
    \end{itemize}

    \textbf{Мотивация выполнения планов:}

    Для мотивации сотрудников ООО "Astadeer" на выполнение планов в установленные сроки, с требуемым качеством и с намеченными затратами планируется использовать \textbf{комплексную систему мотивации}, включающую:

    \begin{itemize}
        \item \textbf{Материальное стимулирование:}
            \begin{itemize}
                \item \textbf{Премиальная часть заработной платы (сдельная премия):}  Сдельная премия для производственного персонала, зависящая от объема и качества выполненной работы, стимулирует производительность и выполнение производственных планов.
                \item \textbf{Премии за достижение KPI для руководителей:}  Премирование руководителей (Коммерческого, Финансового, Технического директоров, Руководителя производства) за достижение установленных KPI, связанных с выполнением планов продаж, финансовых показателей, сроков запуска проектов и других ключевых целей.
                \item \textbf{Система бонусов и поощрений:}  Выплата бонусов за перевыполнение планов, выполнение особо важных или срочных заданий, проявление инициативы и достижение выдающихся результатов.  Поощрение лучших сотрудников по итогам месяца, квартала, года.
            \end{itemize}
        \item \textbf{Нематериальное стимулирование:}
            \begin{itemize}
                \item \textbf{Признание и похвала:}  Публичное признание и похвала достижений сотрудников, выражение благодарности за хорошую работу со стороны руководства.  Отметка лучших сотрудников на общих собраниях и в корпоративных коммуникациях.
                \item \textbf{Возможности для профессионального роста и развития:}  Предоставление сотрудникам возможностей для обучения, повышения квалификации, участия в конференциях и семинарах, карьерного роста внутри компании.
                \item \textbf{Создание комфортной рабочей атмосферы и корпоративной культуры:}  Формирование позитивной и дружелюбной рабочей атмосферы, основанной на взаимном уважении, доверии и командном духе.  Организация корпоративных мероприятий, тимбилдингов и создание условий для комфортной работы и отдыха.
                \item \textbf{Расширение полномочий и делегирование ответственности:}  Предоставление сотрудникам большей самостоятельности и ответственности в работе, делегирование полномочий и вовлечение в процесс принятия решений.  Повышение уровня доверия и ответственности сотрудников за результаты своей работы.
            \end{itemize}
    \end{itemize}
    
